\documentclass{article}
\usepackage{graphicx} 
\usepackage[utf8]{inputenc}    
\usepackage[T1]{fontenc}         
\usepackage{lmodern}            
\usepackage[serbian]{babel}

\title{Baze podataka}
\author{Tanja Šević}
\date{decembar 2025}

\begin{document}
\maketitle
\tableofcontents 
\newpage

\section{Pojam baze podataka}
\indent \textbf{Baza podataka} je organizovana kolekcija podataka. Umesto da informacije čuvamo u svesci ili tabeli na papiru, baza podataka ih čuva u računaru tako da se lako mogu pretraživati, menjati i koristiti. \\
Njena poenta je da olakša pristup podacima, pretraživanje, menjanje njihovog sadržaja ili ažuriranje. \\
\bigskip

\indent \textbf{Baza podataka je strukturirani skup podataka koji se čuva elektronski i kojim upravlja sistem za upravljanje bazama podataka (SGBD), omogućavajući jednostavno čuvanje, pretragu, izmenu i upravljanje informacijama.} \\

\subsection{Zašto su baze podataka važne}
Baze podataka se koriste svuda oko nas:
\begin{itemize}
    \item u školama (spisak učenika i ocena),
    \item u prodavnicama (spisak proizvoda i cena),
    \item u bolnicama (kartoni pacijenata),
    \item na internetu (profili na društvenim mrežama).
\end{itemize}
One pomažu da se informacije brzo pronađu i koriste.

\subsection{Primena baze podataka}
Uzmimo primer iz svakodnevnog života. Kada želimo da pustimo neku pesmu uključimo YouTube i on nam sam na \textit{Glavnoj strani} da predlog pesama koje bismo mogli da pustimo. Da li smo se nekad zapitali na koji način on formira tu listu za nas? \\
U školi postpoji spisak učenika sa njihovim podacima, to je takođe jedna baza podataka.
\bigskip

\subsection{Sistemi za upravljanje bazama podataka}
Najpoznatiji sistemi za upravljanje bazama podataka su:
\begin{itemize}
    \item Oracle Database,
    \item MySQL,
    \item Microsoft SQL Server,
    \item PostgreSQL,
    \item IBM DB2,
    \item Redis,
    \item SQLite.
\end{itemize}
Ti sistemi korisnicima pružaju sve usluge rada sa podacima.

\subsection{Razlika između podataka i informacije}

\subsection{Informacioni sistemi}

\subsection{Veza baze podataka i informacionih sistema}

\subsection{Modeli baza podataka}

\subsection{Relacione baze}

\subsection{Pojam i tipovi ključeva}

\section{Kreiranje baza podataka}

\subsection{Upoznavanje konkretnog sistema za upravljanje baze podataka}

\subsection{Korišćenje unapred kreiranih baza podataka}

\section{Rad s tabelama}
Tabele su osnovni način organizovanja podataka.
\begin{itemize}
    \item Kolona predstavlja jednu vrstu podataka (npr. ime, prezime, ocena).
    \item Red predstavlja jednog učenika ili zapis.
\end{itemize}

\subsection{Kreiranje tabela}
Postoji više načina pomoću kojih možemo kreirati tabelu.
\begin{itemize}
    \item Kreiranje tabele bez čarobnjaka - 
    \item Kreiranje tabele sa čarobnjakom -
    \item Poređenje - 
\end{itemize}

\subsection{Izbor tipa podataka}

\subsection{Postavljanje primarnog ključa}

\subsection{Unos podataka u tabelu}


\section{Veza između tabela}

\subsection{Pojam veze}

\subsection{Kreiranje veze između tabela}

\subsection{Referencijalni integritet}

\section{Pretraživanje i sortiranje}

\subsection{Traženje informacija u tabeli}

\subsection{Sortiranje, filtriranje i indeksiranje}

\end{document}
