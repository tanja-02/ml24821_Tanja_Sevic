\documentclass{article}
\usepackage{graphicx} 
\usepackage[utf8]{inputenc}    
\usepackage[T1]{fontenc}         
\usepackage{lmodern}            
\usepackage[serbian]{babel}

\title{Baze podataka}
\author{Tanja Šević}
\date{decembar 2025}

\begin{document}
\maketitle
\newpage
\tableofcontents 
\newpage

\section{Pojam baze podataka}
\indent \textbf{Baza podataka} je organizovana kolekcija podataka. Umesto da informacije čuvamo u svesci ili tabeli na papiru, baza podataka ih čuva u računaru tako da se lako mogu pretraživati, menjati i koristiti. \\
Njena poenta je da olakša pristup podacima, pretraživanje, menjanje njihovog sadržaja ili ažuriranje. \\
\bigskip

\indent \textbf{Baza podataka je strukturirani skup podataka koji se čuva elektronski i kojim upravlja sistem za upravljanje bazama podataka (SGBD), omogućavajući jednostavno čuvanje, pretragu, izmenu i upravljanje informacijama.} \\

\subsection{Zašto su baze podataka važne}
U savremenom svetu, zahvaljujući napretku tehnologije, količina informacija do kojih možemo doći i kojima možemo raspolagati je velika. 
Danas, u odnosu na greneraciju naših baka i deka, ukoliko želimo da dođemo do određene informacije, na primer, koja hrana ima najviše vlakana, ne moramo tražiti knjigu iz biologije, pa lekciju u kojoj se ta tema obrađuje. 
Dovoljno je Google-u da postavimo pitanje i pretraživač će nam sam dati odgovor.
Ovako brzu pretragu nam omogućavaju upravo \textbf{baze podataka}. One omogućavaju da velika količina informacija bude dostupna u svakom trenutku.
Osim toga, baze podataka te informacije uređuju i skladište. 
\\


\subsection{Primena baze podataka}
Baze podataka se koriste svuda oko nas:
\begin{itemize}
    \item u školama (spisak učenika, predmeta, profesora, ocena, izostanaka),
    \item u prodavnicama (spisak proizvoda i cena),
    \item u bolnicama (kartoni pacijenata, datumi pregleda),
    \item na internetu (profili na društvenim mrežama, onlajn prodavnica, aplikacije za plaćanje računa, sajtovi za puštanje muzike).
\end{itemize}
One pomažu da se informacije brzo pronađu i koriste.
Uzmimo primer iz svakodnevnog života. Kada želimo da pustimo neku pesmu uključimo YouTube i on nam sam na \textit{Glavnoj strani} da predlog pesama koje bismo mogli da pustimo. Da li smo se nekad zapitali na koji način on formira tu listu za nas? 
YouTube koristi baze podataka u kojima se čuvaju informacije o pesmama i izvođačima koje smo prethodno slušali. Na osnovu tih podataka, sistem nam predlaže sadržaj koji procenjuje da će nam se dopasti. Bez baza podataka, YouTube ne bi mogao da zapamti naše izbore i ne bi svaki korisnik imao prilagođenu pretragu.
\\
U školi postpoji spisak učenika sa njihovim podacima, to je takođe jedna baza podataka. 
\bigskip

\subsection{Sistemi za upravljanje bazama podataka}
Najpoznatiji sistemi za upravljanje bazama podataka su:
\begin{itemize}
    \item Oracle Database,
    \item MySQL,
    \item Microsoft SQL Server,
    \item PostgreSQL,
    \item IBM DB2,
    \item Redis,
    \item SQLite.
\end{itemize}
Ti sistemi korisnicima pružaju sve usluge rada sa podacima.

\subsection{Razlika između podataka i informacije}

\subsection{Informacioni sistemi}

\subsection{Veza baze podataka i informacionih sistema}

\subsection{Relacione baze}

\subsection{Pojam i tipovi ključeva}

\section{Kreiranje baza podataka}

\subsection{Upoznavanje konkretnog sistema za upravljanje bazama podataka}

\subsection{Korišćenje unapred kreiranih baza podataka}

\section{Rad s tabelama}
Tabele su osnovni način organizovanja podataka.
\begin{itemize}
    \item Kolona predstavlja jednu vrstu podataka (npr. ime, prezime, ocena).
    \item Red predstavlja jednog učenika ili zapis.
\end{itemize}

\subsection{Kreiranje tabela}
Postoji više načina pomoću kojih možemo kreirati tabelu.
\begin{itemize}
    \item Kreiranje tabele bez čarobnjaka - 
    \item Kreiranje tabele sa čarobnjakom -
    \item Poređenje - 
\end{itemize}

\subsection{Izbor tipa podataka}

\subsection{Postavljanje primarnog ključa}
\textbf{Primarni ključ} je jedinstvena kolona ili kombinacija kolona u tabeli koja jednoznačno određuje svaki red u toj tabeli. 
Osnovna svrha primarnog ključa je da osigura integritet podataka tako što sprečava unos duplih ili \textit{null} vrednosti u ključnoj koloni. Može da se koristi kao referenca u drugim tabelama za uspostavljanje veza između različitih tabela. \\
Kao što je učenicima bitno na kog Marka profesor misli kada kaže da želi da Marko odgovara tako je i u bazi podataka bitna jedinstvenost. 
Uzmimo primer odeljenja u kom postoje dva učenika koji se isto zovu i prezivaju. Potreban nam je autentičan način da ih razlikujemo. 
Upravo zbog te autentičnosti, baze podataka zahtevaju primenu \textbf{primarnog ključa} kako 

\subsection{Unos podataka u tabelu}


\section{Veza između tabela}


\subsection{Pojam veze}
Kada pravimo bazu podataka, to je kao da pravimo veliki digitalni ormar sa fiokama. Svaka fioka (tabela) čuva određene stvari – na primer, u jednoj su podaci o \textbf{učenicima}, a u drugoj o \textbf{knjigama}.
Kada imamo više informacija, moramo da ih povežemo da se ne bismo zbunili. To zovemo \textbf{relacije} ili \textbf{odnosi}.
Evo primera iz škole:
\begin{itemize}
    \item Imamo tabelu \textbf{Učenici} (ime, prezime, razred).
    \item Imamo tabelu \textbf{Knjige} (naslov, pisac).
\end{itemize}
Ako želimo da znamo ko je pozajmio koju knjigu, nećemo stalno ponovo upisivati ime učenika pored svake knjige (to troši prostor i pravi nered). Umesto toga, mi \textit{povežemo} te dve tabele. Tako baza „zna” da je Marko iz 5. razreda uzeo knjigu „Hajdi”.

\subsection{Kreiranje veze između tabela}

\section{Pretraživanje i sortiranje}

\subsection{Traženje informacija u tabeli}

\subsection{Sortiranje i filtriranje}

\end{document}
